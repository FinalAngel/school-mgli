\documentclass[../gruppenarbeit_1.tex]{subfiles}

\begin{document}

\section{Praedikatenlogik}

\textbf{Was sind Praedikate?}
\\
\\Bsp. Aussage über ein mathematisches Objekt: A≔5 ist eine ungerade Zahl
\\
\\Eine Aussage bestand bisher immer daraus, etwas über die Eigenschaft eines Objekts zu sagen. Solche Aussagen werden in der Mathematik «Praedikate» genannt.
\\
\\Die Zuordnung eines Praedikats wird wie folgt dargestellt: P(x). P ist das Praedikat, wobei x das Objekt ist.
\\
\\\textbf{Definintion von verschiedenen Praedikaten}
\\
\\Enthaelt ein Praedikat mehrere Objektvariablen, so nennen wir unser Praedikat «n-stelliges Praedikat» oder eine «n-stellige Aussageform». Wenn jedoch eine Aussage keine Variable enthaelt, reden wir von einem «0-stelligen Praedikat».
\\
\\Bsp.1: n-stelliges Praedikat: Q(x,\ y) := «x ist Automarke und y ist das Modell». Das Praedikat Q(VW,\ Golf) ist wahr, wobei Q(Toyota,\ Polo) falsch ist.
\\
\\Bsp.2: V(3)∶= «3 ist eine ungerade Zahl» ist eine Aussage und somit ein 0-stelliges Praedikat.
\\
\\\textbf{Aequivalenz von zwei Praedikaten}
\\
\\Wenn alle Objektvariablenbelegungen dieselben Wahrheitswerte haben, dann wird das folgendermassen dargestellt: P(x) <==> Q(x) oder P(x) ==> Q(x) und Q(x) ==> P(x)
\\
\\Bsp.: P(x)∶= «x ist durch 20 teilbar» und Q(x)∶= «x ist durch 4 und 5 teilbar». Dann ist P(x) <==> Q(x).
\\
\\Jedoch nicht, wenn: P(x) eine Primzahl ist und Q(x) eine ungerade natuerliche Zahl ist.
\\
\\\textbf{Quantoren}
\\
\\Wenn P irgendein Praedikat ist und x eine Objektvariable einer bestimmten Objektklasse, dann gilt: «für alle x gilt P(x)» die wird folgendermassen geschrieben: Ax:P(x)
\\
\\Das Symbol A wird auch Allquantor oder Generalisator oder Generalisator-Operator genannt.
\\
\\Wenn es mindestens ein x gibt, also: «es gibt (mind.) ein x, so dass P(x) gilt», dann schreibt man: Ex:P(x)
\\
\\Das Symbol E wird Existenzquantor oder Partikularisator genannt.
\\
\\Wenn es jedoch genau nur ein Objekt gibt, dann: «es gibt genau ein x, so dass P(x) gilt». Dies wird wie folgt dargestellt: E!x:P(x).
\\
\\Das Symbol E! wird Eindeutigkeitsquantor oder Einzigkeitsquantor genannt.
\\
\\\textbf{Ausserdem: Ein Quantor bindet staerker als alle anderen logischen Operatoren!!!}
\\
\end{document}